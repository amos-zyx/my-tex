% basic of optimizaiton,优化基础知识
\renewcommand{\b}{\vec{b}}
\newcommand{\x}{\vec{x}}

\begin{proposition}
    给定实对称矩阵 $Q\in\setl{R}^{n\times n}$,向量 $\b\in\setl{R}^n$,
    证明:$\inf_{\x\in\setl{R}^n} \frac{1}{2} \x^T Q\x + \b^T \x > -\infty \iff Q$ 半正定,$\b\in \ima(Q)$.
\end{proposition}

\begin{proof}
    记 $q(\x)\eqdef \frac{1}{2} \x^T Q\x + \b^T \x$. 应该注意到,定义域 $\setl{R}^n$ 是线性空间,
    因此可先考虑 $q(\x)$ 在单位球面 $B \eqdef \set{\x\in\setl{R}^n}{\norm{\x}_2 = 1}$ 上的值,再根据数乘 $t\x\; (t>0)$ 估计\footnote{这里提到的估计方法是,二次项的值关于数乘的倍数 $t$ 的变化速度远大于一次项的变化速度. }二次型在整个定义域内的值. 

    \nextproofphase
    首先说明 $Q$ 应该是半正定的. 假设 $Q$ 不满足半正定,那么存在 $\x_0\in B$,使得 $\frac{1}{2} \x_0^T Q \x_0 < 0$. 
    根据极限
    \begin{equation}
        \lim_{t} q(t\x_0) = \left( \frac{1}{2} \x_0^T Q\x_0 \right) t^2 + \left( \b^T\x_0 \right) t = -\infty,
    \end{equation}
    引出了矛盾,因此 $Q$ 半正定.
    
    \nextproofphase
    在继续证明之前,可以对结论左边进行化简
    \begin{equation}
        \inf_{\x\in\setl{R}^n} q(\x) > -\infty \iff \forall x\in B, \;\text{若}\; \frac{1}{2} \x^T Q\x = 0, \;\text{那么}\; \b^T x = 0.
    \end{equation}
    因此只需证明
    \begin{equation}\label{toprove}
        \text{若}\; \frac{1}{2} \x^T Q\x = 0, \;\text{那么}\; \b^T x = 0 \iff \b\in \ima(Q).
    \end{equation}
    因为 $Q$ 是对称矩阵,它满足 $\ker(Q) = \ima(Q)^{\bot}$. 
    $(\Rightarrow)$,任意选取 $\x\in\ker (Q)$,那么 $Q\x = 0$,根据 \eqref{toprove} 的左边立即有 $\b^T x = 0$,根据正交补的性质,$\b \in \ima (Q)$.
    $(\Leftarrow)$,注意到证明 \eqref{toprove} 的左边时只要求了 $\b\in\ima(Q)$,因此这等价于证明:对于对称半正定矩阵 $Q$,满足关系
    \begin{equation}\label{toprove2}
        \x^T Q\x = 0 \Longrightarrow \x\in\ker(Q).
    \end{equation}
    
    \nextproofphase
    接下来将完成 \eqref{toprove2} 的证明.
    根据正交补空间的性质,$\x$ 可以唯一地被表示
    \begin{equation*}
        \x = \x_1 + \x_2,\quad \x_1\in\ima(Q),\, \x_2\in\ker(Q).
    \end{equation*}
    需要注意到两个事实,
    \begin{enumerate}[nosep]
        \item $\x\in\ker(Q) \iff \x_1 = 0$;
        \item $\x^T Q \x = \left( \x_1 + \x_2 \right)^T Q \left( \x_1 + \x_2 \right) = \left( \x_1 + \x_2 \right)^T Q \x_1 = \x_1^T Q \x_1.$
    \end{enumerate}
    因此要证明 \eqref{toprove2} 等价于证明:对于 $\x_1\in\ima(Q)$,满足关系
    \begin{equation*}
        \x_1^T Q \x_1 = 0 \Longrightarrow \x_1 = 0,
    \end{equation*}
    或者等价于证明它的逆否命题
    \begin{equation}\label{toprove3}
        \x_1 \in \ima(Q)\setminus \{0\} \Longrightarrow \x_1^T Q \x_1 \neq 0.
    \end{equation}
    不妨设 $\ima(Q)\setminus \{0\}$ 非空,取 $\x_1 \in \ima(Q)\setminus \{0\}$. 对称半正定矩阵 $Q$ 存在正的特征值 $\lambda_i > 0\, (1\leq i\leq s)$,记特征值 $\lambda_i$ 对应的特征向量为 $\alpha_{ij}$,
    因此可以写出 $\x_1$ 关于特征子空间的分解 $\x_1 = \sum_{i,j} k_{ij}\alpha_{ij}$ ($k_{ij}$ 不全为 0).
    \begin{equation*}
        \x_1^T Q \x_1 = \left( \sum_{i,j} k_{ij}\alpha_{ij} \right)^T Q \left( \sum_{i,j} k_{ij}\alpha_{ij} \right) 
        = \left( \sum_{i} \sum_{j} k_{ij}\alpha_{ij} \right)^T \left( \sum_{i} \lambda_i \sum_{j} k_{ij}\alpha_{ij} \right)
        = \sum_i \lambda_i \norm{\sum_{j} k_{ij}\alpha_{ij}}^2
        > 0.
    \end{equation*}
    至此已经证明了 \eqref{toprove3} 成立. 
\end{proof}